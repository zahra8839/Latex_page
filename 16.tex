\documentclass{book}

\usepackage[english]{babel}
\usepackage{amssymb}
\usepackage{amsmath}
\usepackage{txfonts}
\usepackage{mathdots}
\usepackage[classicReIm]{kpfonts}
\usepackage[pdftex]{graphicx}


\begin{document}


\noindent \begin{flushright}
\textbf{CHAPTER TWO}

\noindent \textbf{                                                                                                                                                     }

\noindent \textbf{WHAT IS THE NET?}

\noindent \textbf{}

\noindent A mind strectbed to a new idea never returns to its original dimension,

\noindent Oliver Wendell Holmes

\noindent 

\noindent 
\end{flushright}

\noindent \begin{flushleft}
The internet is one of the most compelex machines ever developed. Describing and understanding the internet at all levels of opration has been , and continues to be ,the focus of PH.D.theses of librarians,computer scientist,and electrical enginers. It is not nesseccery for e-researchers to understand the Net on this kind of detailed level of complexity ,but we belive that e-searchers should be knowledgeable enough to disscuss their needs and use of the Net with both technications and potential research subjects .To help gain this larger conceptual understanding and functional vocabulary, this chatpter covers the oprations, we think it is important that researchers understand how to search the Net for information critical to the research process. Thus, we also focus on the techniques and strategies for efficient and effective Net searches.
\end{flushleft}

\noindent \begin{flushright}

\end{flushright}

\noindent \begin{flushleft}
\textbf{WHAT IS THE NET?}

\noindent \textbf{}

\noindent As with many high-tech devevopments, the Internet has both a military and an academic association(Underwired,1997). The Internet had a rather modest and unassuming origin in 1969, during the Cold War ,when the American think tank RAND Corporation and the United States Pentagon joined to designan indestructible information resource (Diamond \& Bates,1995).During this time, the telephone was the primary communication system in use by the military. However ,a problem with the telephone was its dependence on switching stations that could be targeted during an attack. The challenge, then, was to design a communication system that could quickly reroute digital traffic around failed nodes to ensure successful communication after a nuclear war .A strategic solution was possible ,in theory ,through the construction of a datagram network called a catenet and the use of dynamic routing protocols that could

\noindent 
\end{flushleft}


\end{document}

