% Generated by GrindEQ Word-to-LaTeX 
\documentclass{book} % use \documentstyle for old LaTeX compilers

\usepackage[english]{babel} % 'french', 'german', 'spanish', 'danish', etc.
\usepackage{amssymb}
\usepackage{amsmath}
\usepackage{txfonts}
\usepackage{mathdots}
\usepackage[classicReIm]{kpfonts}
\usepackage[pdftex]{graphicx}

% You can include more LaTeX packages here 


\begin{document}

%\selectlanguage{english} % remove comment delimiter ('%') and select language if required


\noindent \begin{flushleft}
constantly and automatically adjust the flow of traffic through the catenet. Essentially ,the network was assumed to be unreliable at all times and was designed to transend its own unreliability. All the nodes in the network were equal in status to all other nodes;

\noindent Each node had its own authority to originate, pass, and receive massages. Nowadays ,massages are divided into packets, and each packet has its own unique address. Every packet begins at a specified source node and ends at another specified destination node.Every packet winds its way through the network on an individual basis. The particular route that each packet takes is not important., only the final destination. Basically, each packet bounces from node to node in the general direction of its destination until it ends up in the proper place(Sterling,1993). With this kind of communication, if a part of the network is destored, massages are automatically routed around the damaged node, resulting in only marginal delays, rather than incapacity of the net.This rather unorganized kind of delivery system might seem inefficient, but it is remarkably sustainable and dependable as a communication system (Diamond \& Bates,1995).

\noindent      RAND Corporation, Massachusetts Institute of Technology (MIT), and the University of California in Los Angeles (UCLA) studied this decentralized, indestructible, packet-switching network during the 1960s. In 1968,the National Physical Laboratory in Great Britain set up the first test network based on these princliples (Sterling,1993),and shortly thereafter, the American Pentagons Advanced Research Project Agency (ARPA) funded a more ambitious project. The switches on the network,were high-speed, but were still in need of a good,stable network. In the fall of 1969, a highspeed network was installed at UCLA and the first long --distance packet-switched network was in operation. By December 1969, there were four nodes on the network, which was named ARPANET , after the American Pentagon sponsor.The four computers on this network could transfer data on dedicated high-speed transmission lines and could even be programmed remotely from other nodes.Scientists and researchers colud easily share information by long distance.In 1971, there were fifteen nodes in ARPANET ; by 1972 there were thirty-seven nodes (Diamond \& Bates, 1995).

\noindent      In 1972-1973 an interesting use of the ARPANET was observed. The main traffic on ARPANET was no longer long-distance computing between scientists collaborating on research projects. Rather, the main traffic was moving from research collaboration and trade notes to `'gossip and schmooze'' (Sterling, 1993). It was not long after this that a ``graduate student hacker attitude took over'' (Diamond \& Bates 1995) and the invention of the mailing list transpired,The mailing list was an ARPANET broadcasting technique where one massage could be sent to large numbers of network subscribes. One of the first and largest mailing list transpired. The mailing list was an

\noindent ARPANET broadcasting technique where one message could be sent to large numbers

\noindent of network subscribers. (Joe of the first and largest mailing lists was "SF-LOVERS"

\noindent For science fiction fans (Sterling). In spite of the disapproval of the ARPANET computer administrators of this kind of non-work-related activity, it not only continued but

\noindent grew ``as if some grim fallout shelter had burst open and a Full-scale Mardi Gras parade

\noindent had come out" $\mathrm{\{}$Sterling in Diamond St Bates$\mathrm{\}}$. By the 1980s, Transmission Control

\noindent Protocol and Internet Protocol $\mathrm{\{}$TCP/IPJ. the software protocol used for packaging

\noindent and addressing messages, was being used by other networks to link to ARPANET,

\noindent including die National Science Foundation $\mathrm{\{}$NSFnet] and Usenet $\mathrm{\{}$a pre-ARPANET

\noindent protocol developed in 1979 to facilitate sharing of news items via UNIX networks$\mathrm{\}}$. It

\noindent was originally called the ARPA-Internet but was eventually shortened to the Internet.
\end{flushleft}


\end{document}

