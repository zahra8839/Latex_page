% Generated by GrindEQ Word-to-LaTeX 
\documentclass{book} % use \documentstyle for old LaTeX compilers

\usepackage[english]{babel} % 'french', 'german', 'spanish', 'danish', etc.
\usepackage{amssymb}
\usepackage{amsmath}
\usepackage{txfonts}
\usepackage{mathdots}
\usepackage[classicReIm]{kpfonts}
\usepackage[pdftex]{graphicx}

% You can include more LaTeX packages here 


\begin{document}

%\selectlanguage{english} % remove comment delimiter ('%') and select language if required


\noindent \begin{flushleft}
$\mathrm{\{}$Diamond 3: Bates. 1995$\mathrm{\}}$. As it progressed. ARPANET divided into two networks:

\noindent military communications on MILNET and computer researchers on ARPANET.

\noindent 

\noindent      In June 1990 ARPANET was decommissioned---"a happy victim of its own over-

\noindent whelming success" (Sterling. 1993$\mathrm{\}}$. Signi?cantly, almost all of the current networks

\noindent use the TCPHP suite of protocols. Vince Cerf. a pioneer on the use and development

\noindent of the Internet for decades. and who also had a tee-shirt made that said IP on everything.

\noindent " told ComputerWorld in 1995. ``I take great pride in the fact that the Internet

\noindent has been able to migrate itself on top of every communications capability invented in

\noindent the past twenty years . . . I think thats not a had achievement"

\noindent  $\mathrm{\{}$Diamond \& Bates,1995).

\noindent 

\noindent The lnternet has evolved into an unstructured network of millions computers

\noindent throughout the world. Today most of us access the Net through the use of a suite of

\noindent programs known as the World Wide Web IWWWEQ that operate at the application

\noindent level above TCPHP. The original concept of the Web and the First server and Web

\noindent browser software were developed in the early 1990s by an Englishman. Tim Berners-

\noindent Lee. Modern Web browsers are computer---based graphical programs that allow the

\noindent user to interface or browse through the Web. Browsers are menu-driven and icon-

\noindent based software that provide a user-friendly tool for accessing the Web. Currently. the

\noindent two most common Web browsers are Internet Explorer and Netscape Communicator;

\noindent both Function through a hyperlinked platform. The term hyperlink is generally used to

\noindent describe the electronic representation of text and graphics that uses the random access

\noindent capabilities of computers to overcome the sequential medium of print on paper $\mathrm{\{}$Marchionini,1998).

\noindent Hypertext has two key elements: nodes and links. The documents or

\noindent units of information are called nodes; the electronic connections to and from the units

\noindent of information are called nodes; the electronic (in Marchionini) points out that ``links: are the

\noindent essence of hypertext since they facilitate jumping from node to node in non-linear

\noindent Fashion" (p.8).

\noindent      The original concept of hypertext can be traced to an American named Vannevar

\noindent Bush. Bush described his vision for a device to help the human mind organize information

\noindent in a landmark article called ``As We May Think" in the July 1945 issue of the

\noindent Atlantic Montbly (December. I994). This device. the memex machine. would allow scientists to systematize and access their related information through an individualized

\noindent associative structure instead of the traditional catalogue anti index cards (Jackson,

\noindent 1997). In Bush`s vision. the memex machine would he a tool For the organization of

\noindent information For scientists. Had the memex machine been built. it would have had an

\noindent interconnected structure that would permit scientists to coordinate and organize their

\noindent increasing information.

\noindent      In 1967. Ted Nelson developed a vision of hypertext similar to that of Bush's.

\noindent Unlike Bush's vision ,however. Nelson envisioned a machine that would be more of a

\noindent tool for the expression and development of ideas than an organizational tool for scientists

\noindent (Jackson, 1997, Nelson, 1967). Nelson's vision included linking information from

\noindent a variety of media such as text. images. and sound. Nelson (in Jackson) also saw:

\noindent 

\noindent [H]ypertext as removing the con?nes of linearity imposed on ideas by existing media.

\noindent In hyper-textual expression. ideas may branch in several directions. and paths through
\end{flushleft}


\end{document}

